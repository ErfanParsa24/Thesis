\chapter{مروری بر ادبیات و کار‌های انجام شده}
\clearpage
هدف های پيش بيني شده و روش ها و ضرورت انجام طرح :
1-	بررسی ترابرد بار و اسپین در بورفین:
بوروفن و گرافن به دلیل داشتن صفحات ضخیم اتمی منفرد شباهت زیادی دارند و دارای ساختار محکم زیادی هستند که انعطاف پذیری فوق العاده ای را به نمایش می گذارد. با این حال ، آنها تمایز قابل توجهی در ساختارهای شبکه خود دارند. بوروفن یک آلوتروپ کریستالی از بور است که به عنوان یک ساختار عمده دارای طبیعت غیر فلزی است. بوروفن ماده ای بسیار ناهمسانگرد است که رفتارهای نیمه رسانا و فلزی را نشان می دهد. اخیراً ، نانوساختار نویدبخش دو بعدی از نظر تئوری پیش بینی شده است و اثرات شکست تقارن سوراخ ذرات توسط هدایت نوری گزارش شده است [46]. علاوه بر این ، اخیراً خواص \lr{ab initio 8-Pmmn borophene} مورد مطالعه قرار گرفته است [47] و \lr{Li} و همکاران. [48] طیف انرژی \lr{8-Pmmn borophene}را بدست آورد و نشان داد که ماده ایده آل برای تجزیه و تحلیل تونل سازی کلاین در حضور ناهمسانگردی است. بعلاوه ، مشابه اثرات میدان مغناطیسی گرافن نیز در این ماده پرداخته شده است [49] اما هنوز هم برای تعیین کاربردهای عملی آن به کارهای زیادی نیاز دارد.

در [50] نشان داده شده است که اولا طیف انرژی الکترونیکی ناهمسانگرد است ثانیا طیف ناهمسانگرد با پتانسیل مثلث گسترش یافته است .
در [50] نشان داده شده‌ است که  ترکیب‌های گوناگون از صفحه‌های بورن وجود دارد و همچنین اثرا بررسی نشده بسیاری باقی‌مانده است که می‌توان آنها را محاسبه کرد.

2-	بررسی ترابرد دره‌ای در صفحات بروفین:
در یک جامد بلوری ، رابطه بین انرژی الکترون و حرکت بلوری آن توسط ساختار باند الکترونیکی ماده اداره می شود. حداقل محلی در باند هدایت یا حداکثر محلی در باند ظرفیت به عنوان یک دره شناخته می شود. علاوه بر شارژ و چرخش ، یک الکترون دارای درجه آزادی دره نیز می باشد که دره ای را که الکترون اشغال می کند مشخص می کند. امکان استفاده از درجه آزادی دره برای ذخیره سازی و حمل اطلاعات (مشابه اسپین در اسپین‌ترونیک) منجر به کاربردهای الکترونیکی مفهومی می شود که به عنوان \lr{valleytronics} [52-53]شناخته می شوند. یک سیستم ماده ولیترونیک ایده آل دارای یک ساختار باند متشکل از دو (یا بیشتر) دره تبهگن اما نابرابر (نقاط اکسترمم طیف انرژی محلی) است که می تواند برای رمزگذاری ، پردازش و ذخیره اطلاعات به‌کار گرفته شود.

ایده بکارگیری درجه آزادی دره الکترونیکی چیز جدیدی نیست. به عنوان مثال ، کاملاً شناخته شده است که می توان از کرنش برای تنظیم انرژی دره ها استفاده کرد ، که یک روش گسترده در صنعت الکترونیک نیمه هادی برای افزایش تحرکات حامل است.[54] در تحقیقات اساسی تر ، نیمه هادی های سنتی ، مانند سیلیکون و آلومینیوم آرسنید \lr{(AlAs)} ، که دارای چندین دره در باند های هدایت واقع در یا در نزدیکی نقاط تقارن \lr{X} در منطقه بریلوین هستند ، مورد بررسی قرار گرفته اند. کارهای اولیه بررسی درجه آزادی دره به دهه 1970 برمی گردد ، با مطالعاتی در مورد لایه های وارونگی در اتصال‌های سیلیکون/عایق [55؛56].

در این مواد 2 بعدی شش ضلعی ، مانند گرافن و تک لایه گروه \lr{VI} که فلزی را ساندویچ کرده است.(\lr{TMD} ها) (به عنوان مثال \lr{MoS2} ، \lr{MoSe2} ، \lr{WS2} و \lr{WSe2}) ، ویژگی های الکترونیکی لبه باند تحت سلطه دو دره ناموزون هستندکه در نقاط \lr{+K} و \lr{-K} در لبه های منطقه بریلوئن  قرار دارند.(شکل 4). این دره ها را می توان با یک شبه اسپین دودویی نشان داد که مانند سیستم \lr{spin-1/2} رفتار می کند. الکترونهای موجود در \lr{+K} می توانند به عنوان شبه ذره اسپین بالا ، و الکترونهای موجود در دره \lr{-K} می توانند به عنوان دراسپین پایین برچسب گذاری شوند. بنابراین ، در یک سیستم دوپینگ ، یک توزیع جمعیت حامل قطبی در یک دره \lr{+K} یا \lr{K} می تواند اطلاعات باینری را ذخیره کند.

اعمال میدان الکتریکی در بخش باریک شکل فوق تراز فرمی را در این بازه جابجا می کند. در اینصورت نمودار نوار انرژی در سه بخش مختلف نوار به شکل زیر در می‌آید:

در مرجع مذکور نشان داده شده است که با بالا بردن تراز فرمی در محدوده کانال، الکترونهایی که دارای اندیس دره ی منفی هستند (دایره های تو خالی) به طور کامل منعکس شده و الکترونهایی که از لید راست خارج می شوند پلاریزاسیون دره ای کامل خواهند داشت. به همین ترتیب اگر میدان الکتریکی منفی اعمال کنیم تراز فرمی کانال پایین تر از تراز فرمی لیدها قرار گرفته و مسیر الکترونهای دارای اندیس مثبت به طور کامل سد می‌شود. (شکل 3)  لذا از این رفتار میتوان برای آشکار سازی جریان پلاریزه دره ای استفاده کنیم.

اثر ولی‌ترونیک در بسیاری از مواد دوبعدی جدید به همراه اثرها و برهمکنش‌های گوناگون بررسی شده است. این مهم در بروفین هنوز به انجام نرسیده است.