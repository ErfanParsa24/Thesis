\chapter{روش پیشنهادی}
\clearpage
روش مود مچینگ :
سازه های موجدار با ناپیوستگی های طولی در بسیاری از کاربردها نقش دارند. تجزیه و تحلیل ناپیوستگی های راهنمای موج از اهمیت بالایی برخوردار است. یک رویکرد قدرتمند ، روش تطبیق حالت \lr{(MMT)} است که با استفاده از آن حالتهای موجی راهنما زمینه ها را توصیف می کنند و سپس شرایط مرزی که باید زمینه های مماس در سطح مقطع ساختار هدایتگر موج داشته باشند ، در رابط اعمال می شوند. بین بخشهای مختلف موجبر. در \lr{MMT} ، ساختار مورد بررسی به زیرسازی های ساده تری تقسیم می شود که حالت های آن (ویژه توابع ) مشخص است یا می توان آنها را تعیین کرد. میدان های الکتریکی و مغناطیسی ناشناخته با مجموع حالت های ویژه با ضرایب ناشناخته تقریبی می شوند. وقتی این عملکردها حالت عادی دارند ، از آنها به عنوان حالت های طبیعی یاد می شود. گسترش سری ، همه شرایط مرزی مشکل را برآورده می کند ، به جز مواردی که در رابط های بین مناطق فرعی مجاور وجود دارد. سپس ضرایب انبساط ناشناخته از شرایط مرزی در رابط ها تعیین می شود.

ترابرد ، دره و اسپین را در یک اتصال سیلیسن طبیعی / فرومغناطیسی / طبیعی در [58]بررسی شده است. یوکوماما نشان داده است که رسانایی بار ، دره و اسپین در این محل اتصال با طول سیلیسن فرو مغناطیسی نوسان می کند. می‌توانیم از آن برای بروفین نیز استفاده کنیم.

ماتریس انتقال و الگوریتم لوپزسانشو برای محاسبه ترابرد
از آنجایی که در توضیح اثرات ابرشبکه بر روی خواص ترابرد ترکیبات دو بعدی از روش ماتریس انتقال استفاده خواهیم کرد در زیر به مرور نحوه کاربرد این مدل برای توصیف سیتمهای ساده تر مثل گرافین تک لایه تحت تأثیر پتانسیل پله ای دوره ای به نحوی که در شکل 1 مشخص گردید، می‌پردازیم.

قسمت بالایی شکل طیف انرژی شبه ذرات را در طول گرافین تک لایه نشان می‌دهد. از آنجایی که انرژی فرمی بین دو نوار گرافینی کنار هم متفاوت است، الگوی پتانسیل سیستم به شکل چاه های پتانسیل متوالی با استفاده از رابطه زیر قابل توصیف است
$$
V(x) = \begin{cases}
    V_0 &\quad l_{(2i-1)< |x| < l_{2i}\quad i=1,2,\dots}\\
    0 &\quad\text{\lr{Otherwise}}
\end{cases}
$$
این شکل پتانسیل مشابه همان شکلی است که در ابرشبکه های مبتنی بر نیمرسانا های معمولی اتفاق می‌افتد. تفاوت اساسی سیستم ما در این خواهد بود که رفتار حامل های بار به جای معادله شرودینگر از معادله دیراک تبعیت می‌کند که همیلتونی آن را می‌توان به شکل زیر نوشت
$$
\hat{H} = -i\hbar v_F\sigma \nabla + V(x)
$$
که در آن $v_f = 106$ \lr{m/s} سرعت فرمی و$\sigma(\sigma_x,\sigma_y)$ماتریس های پائولی هستند. الکترونها و حفره ها در نیمرساناها معمولا توسط دو معادله شرودینگر جداگانه که با یکدیگر ارتباطی ندارند توصیف می‌شوند. اما در سیستم های دو بعدی نظیر این سیستم الکترونها و حفره ها رفتار مرتبط با یکدیگر دارند و دارای خصوصیات کایرال می‌باشند. به این ترتیب که با یک تابع موج دو مؤلفه ای (توابع اسپینور) توصیف می‌شوند.

برای محاسبه ترابرد سیستمی که با همیلتونی رابطه 2 توصیف می‌شود، با در نظر گرفتن پاسخ کلی به صورت موج تختی که با زاویه معینی در جهت ابرشبکه حرکت می‌کند، مؤلفه های اسپینور دیراک بدست می‌آید. سپس با اعمال شرط پیوستگی تابع موج در مرز بین نوارهای مختلف ماتریس انتقال در اینترفیس دو ناحیه مجاور بدست می‌آید. از حاصلضرب ماتریس های انتقال ماتریس انتقال ابرشبکه به صورت زیر حاصل می‌شود:
\begin{equation}
    \begin{split}
    S(z) = S(z=l_{2}) = S(z=l_{3}) &= \dots S(z=l_{i})       \dots S(z=l_{n-1})\\
    S(z=l_{i}) &= 
        \begin{bmatrix} 
            t_{11} & t_{12} \\
            t_{21} & t_{22}
        \end{bmatrix}
    \end{split}
\end{equation}

به این ترتیب احتمال عبور موج به صورت تابعی از زاویه برخورد بدست می‌آید که برای این سیستم رابطه زیر بدست می‌آید:
\begin{equation}
    T(\phi) = \frac{e^{-2ik_x\prime D-q(D+L)(1+e^{2i\theta})^2(1+e^{2i\phi})^2}}{A_1^2 + B_1^2}
\end{equation}
پس از  بدست آمدن ضرایب عبور می‌توان با استفاده از رابطه بوتیکر رسانش سیستم را بدست آورد:
\begin{equation}
    G = G_0 \int_{-\pi/2}^{\pi/2} T(E,\sqrt{2E} sin\phi)cos\phi d\phi
\end{equation}
که در آن$G_0 = e^2m v_F w/\hbar^2$خواهد بود.

روش لوپز-سانشو برای محاسبه ترابرد در سیستمهایی که در یک بعد شرط مرزی بلاخ را رعایت می‌کنند دارای همگرایی بالایی است و توان محاسباتی کمی نیاز دارد. برای توضیح روش فرض می‌کنیم یک نانونوار گرافینی در یک بعد دارای ساختار تناوبی می‌باشد. ابتدا باید به کمک هامیلتونی \lr{TB} ماتریس برهمکنش یک سوپرسل با خودش $H_{0,0}$ را محاسبه و همچنین برهمکنش یک سوپرسل با سلول مجاورش$H_{0,1}$ را محاسبه کنیم.


حال به کمک ماتریس هاي فوق و روش لوپز خود انرژي الکترودهاي طرفین را محاسبه می‌کنیم. براي این کار ابتدا توابع گرین سطحی الکترودها را محاسبه می‌کنیم.
\begin{equation}
    \begin{split}
        g_{0,0}^L = \left[ E^{+}I - H_{0,0} -H^{\dagger}_{-1,0} \right]^{-1}\\
        g_{M+1,M+1}^R = \left[ E^{+}I - H_{0,0} -H^{\dagger}_{-1,0} \right]^{-1}
    \end{split}
\end{equation}
در روابط فوق $E^{+} = E + i\eta$  می‌باشدکه $\eta$ مقدار کوچکی است که برای فرار از تکینگی به کار می‌بریم. برای استفاده از روابط فوق نیاز داریم ماتریس های انتقال را محاسبه کنیم. برای این کار از یک روش تکرار استفاده می‌شود.
\begin{equation}
    \begin{split}
        \Lambda = t_0 + \hat{t_0} t_1 + \hat{t_0}\hat{t_1}t_2 + \dots \hat{t_0}\hat{t_1}\hat{t_2}\dots t_n\\
        \hat{\Lambda} = \hat{t_0} + t_0 \hat{t_1} + \hat{t_0}\hat{t_1}t_2 + \dots t_0 t_1 t_2\dots \hat{t_n}
    \end{split}
\end{equation}
که در آن $t_i$ و $\hat{t_i}$  به نحو زیر تعریف می‌شوند:
\begin{equation}
    \begin{split}
        t_i = \left(I-t_{i-1}\hat{t}_{i-1}-\hat{t}_{i-1}t_{i-1}\right)^{-1} t_{i-1}^2\\
        \hat{t_i} = \left(I-t_{i-1}\hat{t}_{i-1}-\hat{t}_{i-1}t_{i-1}\right)^{-1} \hat{t}_{i-1}^2
    \end{split}
\end{equation}
و به ازای مقدار $i = 0$ داریم:
\begin{equation}
    \begin{split}
        t_0 = \left(E^{+}I-H_{0,0}\right)^{-1} H^{\dagger}_{-1,0}\\
        t_0 = \left(E^{+}I-H_{0,0}\right)^{-1} H_{-1,0}
    \end{split}
\end{equation}
با در نظر گرفتن سوپرسل نمونه به عنوان قسمتی از لید چپ به صورت لایه به لایه ( از $l = M$ تا $l = 2$) می‌توانیم تابع گرین سطحی را بدست آوریم:
\begin{equation}
    g_{l,l}^R = \left[E^+ I- H_{l,l} - H_{l,l}g_{l+1,l+1}^RH^{\dagger}_{l,l+1} \right]^{-1}
\end{equation}
پس از آن تابع گرین کل از رابطه زیر قابل محاسبه خواهد بود:
\begin{equation}
    G_{11} = \left[E^+I- H_{1,1}- \Sigma^L-\Sigma^R\right]
\end{equation}
که در آن: 
\begin{equation}
    \begin{split}
        \Sigma^L = H^{\dagger}_{0,1} g^{L}_{0,0}H_{0,1}\\
        \Sigma^R = H_{1,2} g^{R}_{2,2}H^{\dagger}_{1,2}
    \end{split}
\end{equation}
خود انرژیهای مربوط به برهمکنش با لیدهای چپ و راست هستند. از تابع گرین می‌توان چگالی حالتهای جایگزیده \lr{(LDOS)} را نیز بدست آورد: 
\begin{equation}
    n_j = - \frac{1}{\pi} Im[G_{(j,j)}]
\end{equation}
که اندیس $j,j $به سایت مورد نظر اشاره می‌کند. در نهایت با استفاده از فرمول لانداور می‌تواند رسانش را بدست آورد:
\begin{equation}
    G_{(E)} = \frac{2e^2}{h} T_{(E)}
\end{equation}
که در آن از روابط زیر برای محاسبه احتمال عبور استفاده شده است: 
\begin{equation}
    \begin{split}
        T_{(E)} = Tr[\Gamma^{L}G_{11}\Gamma^{R}G^{\dagger}_{1,1}]\\
        \Gamma^{L,R} = i [\Sigma^{L,R}- (\Sigma^{L,R})^{\dagger}]
    \end{split}
\end{equation}
به این ترتیب هزینه محاسباتی این سیستم  $2N x 2N$ به طور خطی با $N$ تغییر می‌کند. این روش را به راحتی می‌توان با تنظیم پارامترهای هاپینگ و آنسایت به ابر شبکه ها تعمیم داد به طوری که سوپر سل حاوی یک دوره تناوب ابر شبکه باشد. 
