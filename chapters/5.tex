\chapter{نتیجه گیری}

\clearpage
\section{مقدمه:}
از آنجایی که روند کوچک‌سازی پردازشگرها در نزدیکی نهایت خود قرار گرفته است، بنابراین استفاده از مواد جدید که می‌توانند در خط تولید \gls{CMOS} نیز تولید شوند بسیار مورد توجه قرار گرفته‌اند. 
حضور مواد دوبعدی مثل گرافین و دیگر ورقه‌های دوبعدی امیدهای زیادی برای شکستن سد نهایی کوچک‌سازی پردازشگرها، زنده کرده است. 
اگرچه به مدت کوتاهی از تولید گرافین مواد دوبعدی دیگری نیز تولید شده‌اند. در میان تمام آنها بورفین خواص منحصر به فردی از خود نشان داده است. بورفین در خواص مختلفی برتری‌های نسبی بردیگر مواد دارد.
در این رساله به استفاده از بورفین ساخت وسایل الکترونیکی آینده پرداخته شده است. 
پتانسیل‌ها، تفاوت‌ها، محدودیت‌ها و برتری‌های آن در ساخت چند دستگاه الکترونیکی که بسیار پرکابرد هستند و در منطق و نحوه ساخت پردازشگرهای آتی نفش خواهند داشت بررسی شده است.

ترابرد بار و اسپین از مهمترین خصایصی است که در هر ماده جدید دوبعدی باید مورد بررسی قرار بگیرد چرا که تقریبا بار تمام دستگاه‌های الکترونیکی برای پردازش بردوش آنهاست.
از طرفی نباید به حضور ناحالصی‌ها، نقص‌ها و اشکالات دیگر را نیز نادیده گرفت. بر همین مبنا در ابتدا ترابرد بار در بروفین $\beta_{12}$ در حضور انواع نقص‌ها از عیوب نقطه‌ای تا عیوب کلی تر مثل پراکندگی اندرسون بررسی شده است. سپس دستگاه‌های جدید همانند شیر‌اسپینی که ترابرد بار و اسپین با حضور میدان مغناطیسی در الکترودهاست، بررسی می‌شود و اثرات جانبی آن مانند مغناطومقاوت و گشتاور انتقال اسپینی نیز مورد بحث قرار می‌گیرد.
\section{بررسی ترابرد بوروفین در حضور انواع نقص‌ها:}

در این رساله، رسانایی نانوروبان‌های ‌بوروفین زیگزاگ و آرمچیر با نقص‌های منفرد و اختلالات ضعیف در لبه‌ها را در چارچوب مدل تنگ‌بست با استفاده از تکنیک تابع گرین بررسی کرده‌ایم. رسانایی نانوروبان‌های بوروفین خالص رفتار پله‌ای را نشان می‌دهد و تعداد کانال‌های رسانایی در انرژی فرمی به عرض ‌بوروفین بستگی دارد، به دلیل وجود اتم مرکزی در سلول واحد بوروفین، در تضاد واضح با مورد گرافین. ابتدا دو نوع نقص لبه‌ای را معرفی می‌کنیم، یعنی نقص‌های منفرد شامل تک پراکندگی و تک جای‌خالی و اختلال ضعیف. نانوروبان‌های بوروفین زیگزاگ همگی رفتار فلزی از خود نشان می‌دهند، در حالی که \lr{BNR} های ارمچیر بسته به عرض، رفتار فلزی یا نیمه‌رسانا را نشان می‌دهند. مشخص شده است که عیوب منفرد لبه‌ای باعث حالت‌های شبه موضعی می‌شوند و بنابراین افت رسانایی ظاهر می‌شوند. بنابراین، این نتایج ممکن است وجود شبکه لانه زنبوری پنهان را در داخل یک $\beta_{12}$ ‌بوروفین با وجود ساختار شبکه مستطیلی آن تأیید کند. علاوه بر این، جای خالی یک اتمی و دو اتمی بر انتقال الکترون‌ها به طور قابل توجهی نسبت به مورد قبلی از طریق تشکیل حالت های شبه موضعی و افت رسانایی تأثیر می گذارد. در \lr{BNR} های صندلی راحتی، جای خالی یک اتمی باعث شکسته شدن تقارن زیرشبکه می شود و بنابراین رسانایی را مهمتر از جای خالی دو اتمی تحت تاثیر قرار می دهد، در حالی که در \lr{BNR} های زیگزاگ، جای خالی دو اتمی تقارن زیرشبکه را شکسته و انتقال را بیش از جای خالی یک اتمی تحت تاثیر قرار می دهد. توزیع تصادفی پراکندگی ضعیف روی لبه‌های \lr{BNR} باعث می‌شود که محلی‌سازی اندرسون اتفاق بیفتد، که منجر به انتقال فلز به نیمه‌رسانا می‌شود. در \lr{BNR} ها، طول جایگزیدگی اندرسون برای استحکام پتانسیل پایین بسیار طولانی است که با افزایش قدرت پتانسیل کاهش می یابد. با افزایش طول نانوروبان، رسانایی تا حد زیادی کاهش می یابد. در \lr{BNR} های صندلی راحتی، پتانسیل‌های استحکام ضعیف باعث طول کوتاه تری نسبت به جایگزیدگی اندرسون در مقایسه با \lr{BNR} های زیگزاگ می شود. علاوه بر این، طول جایگزیدگی به عرض \lr{BNR} ها بستگی دارد به طوری که با افزایش عرض \lr{BNR} ها، طول جایگزیدگی نیز افزایش می یابد.

\section{بررسی ترابرد اسپینی:}
در نتیجه، تحقیق ارائه شده در این رساله به بررسی پتانسیل ‌بورفین به عنوان ماده پایه برای شیرهای اسپین در \gls{spintronic} ‌‌می‌‌پردازد. مزایای ‌بورفین در مقایسه با سایر مواد دوبعدی برجسته شده است، از جمله کارهایی مانند چگالی کم، سختی بالا، مقاومت در برابر حرارت و رسانایی الکتریکی. تجزیه و تحلیل نظری ترابرد وابسته به اسپین و گشتاور ترابرد اسپین در اتصالات فرومغناطیس/نرمال/فرومغناطیس مبتنی بر ‌بورفین، نتایج ا‌میدوارکنندهای را نشان ‌می‌دهد، از جمله فلات‌های مقاومت مغناطیسی کامل در تمام محدوده‌های انرژی، شکاف رسانایی در پیکربندی \gls{AP} بدون شکاف نوار به دلیل قانون انتخاب پاریته، فلات‌های کامل \gls{MR} نه تنها در اطراف انرژی نقطه دیراک (مانند گرافین و سیلیسین و غیره) رخ ‌‌می‌‌دهد، بلکه در تمام محدوده‌های انرژی رخ ‌‌می‌‌دهد که محدودیت‌های انرژی‌های خاص را برطرف ‌‌می‌‌کند، یک الگوی نوسانی سینوسی در گشتاور ترابرد اسپین، و تنوع و پراکندگی بیشتر. منحنی \lr{STT-E} به احتمالات بیشتری برای کنترل جداسازی اسپین پیک ‌می‌کند. این نتایج محاسبه شده پتانسیل ‌بورفین را به عنوان یک ماده ا‌میدوارکننده برای دستگاه‌های \gls{spin valve} در \gls{spintronic} نشان ‌می‌دهد. یکی از یافته‌های که الکترودی، مشاهده فلات‌های مقاومت مغناطیسی کامل در ‌بورفین است که نشان دهنده پتانسیل آن به عنوان یک کاندیدای عالی دریچه اسپین است. نمودار رسانایی صفرهای بیشتری را در پیکربندی پاد‌موازی نشان ‌می‌دهد که امکان کنترل دقیقتر دستگاه را در ایجاد حالت‌های روشن و خاموش برای کلید فراهم ‌می‌کند. این ویژگی باعث ‌‌می‌‌شود که ‌بورفین یک کاندید عالی برای شیرهای اسپینی باشد. علاوه بر این، الگوی نوسانی سینوسی در گشتاور ترابرد اسپین (\gls{STT}) در زوایای نسبی مختلف بین الکترودهای مغناطیسی با افزایش شدت مغناطیسی تقویت ‌می‌شود و وسیله‌ای برای دستکاری اسپین در دستگاه فراهم ‌می‌کند. مقایسه با سایر مواد دوبعدی مانند گرافین و سیلیسن رفتار منحصر به فرد ‌بورفین را از نظر رسانایی و \gls{STT} نشان ‌‌می‌‌دهد. \gls{STT} در ‌بورفین یک رفتار سینوسی با‌هارمونیک‌های بیشتر در مقایسه با مواد دیگر نشان ‌‌می‌‌دهد. علاوه بر این، نمودار \gls{STT} در مقابل انرژی فر‌‌می‌‌ و زوایای نسبی الکترودها پیک‌هایی را نشان ‌می‌دهد که متناسب با تفاوت بین ترابرد اسپین بالا و اسپین پایین است. وضوح پیک‌ها نشان دهنده اختلاف بیشتر در ترابرد است، در حالی که نقاط نزدیک به صفر \gls{STT} نشان دهنده تفاوت بسیار ک‌‌می‌‌است. \gls{STT} همچنین به زاویه نسبی بین الکترودها بستگی دارد، با صفر \gls{STT} در زوایایی که مضربی از $\theta$ هستند. به طور کلی، تحقیقات بر روی شیرهای اسپینی مبتنی بر ‌بورفین، بینش‌های ارزشمندی را در مورد پتانسیل این ماده در \gls{spintronic} ارائه ‌‌می‌‌دهد. خواص و رفتار منحصر به فرد ‌بورفین آن را به یک نامزد ا‌میدوارکننده برای دستگاه‌های \gls{spin valve} آینده تبدیل ‌‌می‌‌کند. کاوش و توسعه بیشتر در این ز‌مینه به پیشرفت فناوری \gls{spintronic} و کاربردهای آن در حافظه‌های غیر فرار و دستگاه‌های منطقی کمک خواهد کرد.

% \section{مغناطومقاومت:}

% \section{بررسی ترابرد گشتاور انتقال اسپینی:}


\section{پیشنهادات:}
