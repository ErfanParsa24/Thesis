\chapter{فصل مقدمه} 
\pagenumbering{arabic}
\newpage
در دو دهه گذشته پیشرفت های عظیمی صورت گرفته است که به طور موثری اسپینترونیک و مغناطیسی را به یک نیروی قدرتمند که حوزه دستگاه های حافظه را شکل می دهد، ترکیب کرده است. مواد و پدیده‌های جدید با سرعت چشمگیری کشف می‌شوند و مجموعه‌ای رو به افزایش از بلوک‌های ساختمانی را فراهم می‌کنند که می‌توانند در طراحی دستگاه‌های کاربردی ترانزیستور مانند آینده مورد استفاده قرار گیرند. هدف این مقاله ارائه یک پایه کمی برای این رویکرد بلوک ساختمانی است، به طوری که اکتشافات جدید را می توان در مفاهیم دستگاه عملکردی ادغام کرد، به سرعت تجزیه و تحلیل و ارزیابی انتقادی کرد. از طریق معیارهای دقیق در برابر تئوری و آزمایش موجود، مجموعه‌ای از ماژول‌های عنصری را ایجاد می‌کنیم که مواد و پدیده‌های متنوع را نشان می‌دهند. این ماژول‌های عنصری را می‌توان به طور یکپارچه برای مدل‌سازی دستگاه‌های کامپوزیتی که شامل پدیده‌های اسپینترونیک و نانومغناطیسی است، ادغام کرد. ما تصور می کنیم که کتابخانه ماژول ها هم با ترکیب ماژول های جدید و هم با بهبود ماژول های موجود با پیشرفت زمینه تکامل یابد. سهم اصلی این مقاله ایجاد قوانین یا پروتکل‌های پایه برای یک رویکرد مدولار است که می‌تواند یک پل پایدار بین دانشمندان مواد و طراحان مدار در زمینه اسپینترونیک و نانومغناطیسی ایجاد کند.
تحولات دو دهه اخیر زمینه های متمایز اسپینترونیک و مغناطیس را در یک نیروی قدرتمند ترکیب کرده است. با شروع اثر مقاومت مغناطیسی غول پیکر (GMR)، این میدان با اکتشافات جدید مانند اثر مقاومت مغناطیسی تونل بزرگ (TMR)، سوئیچینگ اسپین-انتقال-گشتاور (STT) و اخیراً، پدیده های مدار اسپینی بالا از جمله اثر سالن چرخش غول پیکر (GSHE) و عایق های توپولوژیکی (TI).
دستگاه های حافظه اسپینترونیک مبتنی بر TMR و STT قبلاً تجاری شده اند در حالی که دستگاه های منطقی اسپینترونیک هنوز به طور فعال در حال بررسی هستند. مواد و پدیده های جدید با سرعت چشمگیری کشف می شوند که می توانند به عنوان مجموعه ای از "بلوک های ساختمانی" به طور مداوم در حال گسترش برای دستگاه های کاربردی پیچیده در نظر گرفته شوند.

گرافن اولین ماده دوبعدی کشف شده است [1]. کشف خصوصیات حیرت انگیز گرافن مجموعه ای از مواد جدید را به وجود آورده است که به عنوان "مواد دو بعدی" شناخته می شوند [2-4]. فرم های دو بعدی برای بسیاری از کاربردها یک منطقه نسبتاً هیجان انگیز و جدید است [5 ، 6]. معمولاً مواد دو بعدی دارای بسیاری از خصوصیات فیزیکی برجسته هستند که برای دستگاه های الکترونیکی ، مهندسی نانو ، تبدیل انرژی و فوتونیک امیدوار کننده است [7–11]. با توسعه سریع گرافن ، اخیراً مواد دو بعدی ، مانند فسفرن ، 
\lr{BN} ، ژرمن ، آنتیمونن ، سیلیسن ، آرسنن و دی الكوژنیدهای فلزات انتقالی ، مورد توجه گسترده قرار گرفته اند.

توده ای از مواد با ضخامت اتم از نظر تئوری پیش بینی یا سنتز شده است [12–16]. با کمال تعجب ، آنها ساختارهای متفاوتی از گرافن دارند ، این اختلاف از درجه های متفاوت خمیدگی حاصل می‌شود [17]. علاوه بر مواد دو بعدی لایه برداری شده از نمونه های بالک، برخی از مواد دوبعدی نیز می توانند از مواد بالک بدون فرم لایه ای تولید شوند ، مانند ترکیب بور تخت دوبعدی  \lr{GaN 2D} و هافنن [18 ، 19].

تحقیقات در مورد بور در ترکیبات مختلف را می توان به چند صد سال پیش بازگرداند ، زیرا بور دارای خاصیت فوق العاده ای است که می تواند تقریباً با تمام عناصر دیگر ترکیب شود.


در میان آنها ، نیترید بور شش ضلعی \lr{(h-BN)} یک ترکیب بند وسیع \lr{III-V} است. این یک ماده لایه ای با ساختار گرافیت مانند است که در آن شبکه های مسطح شش ضلعی \lr{h-BN} مرتباً روی هم انباشته می شوند. \lr{h-BN} دارای پایداری شیمیایی بالا ، خصوصیات فیزیکی عالی و هدایت حرارتی بالایی است [20–23].

در سال 2015 ، ورق بور \lr{2D} با موفقیت بر روی بسترهای نقره  \lr{(Ag)} ساخته شد [25]. مطالعه بوروفن محققان زیادی را در بسیاری از زمینه ها مانند علوم مواد ، فناوری نانو ، فیزیک ، شیمی و مواد تغلیظ شده جذب کرده است [13 ، 26 ، 27]. \lr{"Borophene"} نانو صفحه جدید بور با ضخامت اتم برای سنتز در مقیاس بزرگ است [28]. این سبک ترین ماده دوبعدی تا به امروز است. بوروفن همسایه گرافن است و بنابراین ، داشتن برخی از خصوصیات مشابه گرافن مطلوب است [29].

هر دو الکترون $\sigma$ و $\pi$ در بوروفن حالات الکترونیکی سطح فرمی را اشغال می کنند و آن را ابررسانا می کنند. فشار زیاد و فشار خارجی وجود ندارد. بوروفن می تواند بالاترین \lr{Tc} را در بین مواد \lr{2D} داشته باشد. برای ساختارهای بور\lr{2D} ، پیچیدگی شیمیایی و ساختاری ، خصوصیات الکترونیکی و پایداری به طور گسترده مورد بررسی قرار گرفته است [27 ، 30 ، 31].

اخیراً ، یک نانوساختار کاملاً فلزی مبتنی بر بور بر روی یک کریستال نقره با رسوب بخار فیزیکی ، به نام بوروفن، ساخته شده است [32–36]. فازهای زیادی از آلوتروپهای بور بالک و دوبعدی مانند $\alpha$ ، $\beta$ و غیره وجود دارد که از لحاظ نظری پیشنهاد شده اند [37–39]. 

اگرچه ، بور از مشارکت در تشکیل پیوندهای شیمیایی برای ایجاد یک شبکه لانه زنبوری پایدار جلوگیری می کند ، اما ممکن است یک ساختار مسطح پایدار توسط مخلوطی از لانه زنبوری همراه با واحدهای مثلثی ایجاد شود [42،43] این ساختار شامل دو اتم در سلول واحد اولیه است که \lr{2B:Pmmn} نامیده می شود، که در آن \lr{Pmmn} مخفف گروه فضایی 59 است که در یک سیستم بلوری \lr{orthorhombic} وجود دارد. \lr{Xu} و همکاران [44] یک ماده جدید دیراک را پیش بینی کرد: بوروفن هیدروژنه (بوروفان) ، مشخصه‌های دیراک را با سرعت قابل توجه فرمی نشان می دهد که تقریبا دو برابر گرافن است.

\section{ساختار رساله:}