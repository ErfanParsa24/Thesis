\chapter{فصل مقدمه} 
\pagenumbering{arabic}
\newpage
گرافن اولین ماده دوبعدی کشف شده است [1]. کشف خصوصیات حیرت انگیز گرافن مجموعه ای از مواد جدید را به وجود آورده است که به عنوان "مواد دو بعدی" شناخته می شوند [2-4]. فرم های دو بعدی برای بسیاری از کاربردها یک منطقه نسبتاً هیجان انگیز و جدید است [5 ، 6]. معمولاً مواد دو بعدی دارای بسیاری از خصوصیات فیزیکی برجسته هستند که برای دستگاه های الکترونیکی ، مهندسی نانو ، تبدیل انرژی و فوتونیک امیدوار کننده است [7–11]. با توسعه سریع گرافن ، اخیراً مواد دو بعدی ، مانند فسفرن ، 
\lr{BN} ، ژرمن ، آنتیمونن ، سیلیسن ، آرسنن و دی الكوژنیدهای فلزات انتقالی ، مورد توجه گسترده قرار گرفته اند.

توده ای از مواد با ضخامت اتم از نظر تئوری پیش بینی یا سنتز شده است [12–16]. با کمال تعجب ، آنها ساختارهای متفاوتی از گرافن دارند ، این اختلاف از درجه های متفاوت خمیدگی حاصل می‌شود [17]. علاوه بر مواد دو بعدی لایه برداری شده از نمونه های بالک، برخی از مواد دوبعدی نیز می توانند از مواد بالک بدون فرم لایه ای تولید شوند ، مانند ترکیب بور تخت دوبعدی  \lr{GaN 2D} و هافنن [18 ، 19].

تحقیقات در مورد بور در ترکیبات مختلف را می توان به چند صد سال پیش بازگرداند ، زیرا بور دارای خاصیت فوق العاده ای است که می تواند تقریباً با تمام عناصر دیگر ترکیب شود.


در میان آنها ، نیترید بور شش ضلعی \lr{(h-BN)} یک ترکیب بند وسیع \lr{III-V} است. این یک ماده لایه ای با ساختار گرافیت مانند است که در آن شبکه های مسطح شش ضلعی \lr{h-BN} مرتباً روی هم انباشته می شوند. \lr{h-BN} دارای پایداری شیمیایی بالا ، خصوصیات فیزیکی عالی و هدایت حرارتی بالایی است [20–23].

در سال 2015 ، ورق بور \lr{2D} با موفقیت بر روی بسترهای نقره  \lr{(Ag)} ساخته شد [25]. مطالعه بوروفن محققان زیادی را در بسیاری از زمینه ها مانند علوم مواد ، فناوری نانو ، فیزیک ، شیمی و مواد تغلیظ شده جذب کرده است [13 ، 26 ، 27]. \lr{"Borophene"} نانو صفحه جدید بور با ضخامت اتم برای سنتز در مقیاس بزرگ است [28]. این سبک ترین ماده دوبعدی تا به امروز است. بوروفن همسایه گرافن است و بنابراین ، داشتن برخی از خصوصیات مشابه گرافن مطلوب است [29].

هر دو الکترون $\sigma$ و $\pi$ در بوروفن حالات الکترونیکی سطح فرمی را اشغال می کنند و آن را ابررسانا می کنند. فشار زیاد و فشار خارجی وجود ندارد. بوروفن می تواند بالاترین \lr{Tc} را در بین مواد \lr{2D} داشته باشد. برای ساختارهای بور\lr{2D} ، پیچیدگی شیمیایی و ساختاری ، خصوصیات الکترونیکی و پایداری به طور گسترده مورد بررسی قرار گرفته است [27 ، 30 ، 31].

اخیراً ، یک نانوساختار کاملاً فلزی مبتنی بر بور بر روی یک کریستال نقره با رسوب بخار فیزیکی ، به نام بوروفن، ساخته شده است [32–36]. فازهای زیادی از آلوتروپهای بور بالک و دوبعدی مانند $\alpha$ ، $\beta$ و غیره وجود دارد که از لحاظ نظری پیشنهاد شده اند [37–39]. 

اگرچه ، بور از مشارکت در تشکیل پیوندهای شیمیایی برای ایجاد یک شبکه لانه زنبوری پایدار جلوگیری می کند ، اما ممکن است یک ساختار مسطح پایدار توسط مخلوطی از لانه زنبوری همراه با واحدهای مثلثی ایجاد شود [42،43] این ساختار شامل دو اتم در سلول واحد اولیه است که \lr{2B:Pmmn} نامیده می شود، که در آن \lr{Pmmn} مخفف گروه فضایی 59 است که در یک سیستم بلوری \lr{orthorhombic} وجود دارد. \lr{Xu} و همکاران [44] یک ماده جدید دیراک را پیش بینی کرد: بوروفن هیدروژنه (بوروفان) ، مشخصه‌های دیراک را با سرعت قابل توجه فرمی نشان می دهد که تقریبا دو برابر گرافن است.

ب )  اصول و فرضيات :
یک مخروط دیراک در ساختار نواری بوروفن هیدروژنه (بوروفان) بین نقاط گاما و ایکس دیده می‌شود، در مرجع[41].با مقایسه معادله دیراک برای ذرات با اسپین نیمه صحیح و هامیلتونی موثر بی جرم دو نوار انرژی صفحه که در انرژی های پایین معتبر است. همیلتونین مربوط به مخروط دیراک واقع در $k_d = (0.64،0،0 ±)^{-1}$\AA با معادله زیر آورده شده است:
$$
H_D = v_x \sigma_x p_x + v_y \sigma_y p_y + v_t I p_x
$$

جایی که $\sigma_x$ و $\sigma_y$ ماتریس های پاؤلی هستند و $I$ ماتریس یکانی به بعد 2 است. این عبارت یک مخروط دیراک \lr{2D} ناهمسانگرد کلی را تعریف می کند ، که با سه ثابت $v_x$ ، $v_y$ , $v_t$ توصیف می شود ، که به معنای سرعت در $x$ و $y$ است جهت و درجه شیب به ترتیب. قطری‌سازی این نتایج همیلتونین در پراکندگی انرژی است
$$
\epsilon(k_x,k_y) = (k_x - k_y) \pm \sqrt{(kx_ -k_y)^2 v_x^2 + k_y^2 v_y^2}
$$

طوری که $v_x = 19.58\times 10^5$ \lr{m / s} ، $v_y = 6.32\times 10^5$ \lr{m / s} ، $v_t = -5.06\times 10^5$ \lr{m / s}. یک نمودار کانتور از مخروط ناهمسانگرد دیراک در شکل 4 نشان داده شده است. سرعت در جهات $x$ مثبت و منفی به ترتیب توسط $v_x + | v_t | = 24.64 \times 10^5$  \lr{m / s} و $v_x - | v_t | = 14.52 \times 10^5$ متر در ثانیه $v_x + | v_t | ، v_x - | c_t|$ ، و $v_y$ برابر با $2.95$ ،$ 1.74$ و $0.76$ برابر سرعت فرمی گرافن هستند ($v_f =8.36\times 10^5$ \lr{m/s} ).[41]

روش ماتریس انتقال مورد بررسی قرار گرفته است [45] در این تحقیق تلاش خواهیم کرد که نتایج استفاده از روش تابع گرین غیر تعادلی که در بخش پایانی به آن اشاره شده است را برای این سیستم بررسی نماییم. می توانیم با استفاده از مدل تنگ بست با اضافه کردن اثرات مختلف به همیلتونی از طریق الگوریتم محاسباتی لوپز-سانشو اثر آن را روی ترابرد بررسی نماییم.
\index{مثال}