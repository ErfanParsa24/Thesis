\addtocontents{toc}{
    \protect\renewcommand\protect\cftchappresnum{\appendixname~}
    \protect\setlength{\cftchapnumwidth}{\mylenapp}}
   
\chapter{ضمایم}%\label{appendices}
\hypertarget{appendices}{}
\thispagestyle{empty}
\clearpage
\indent{
در این قسمت اطلاعات مربوط به توزیع اندازه‌ی افراز‌های تک تک ۲۲ گراف مورد آزمایش که مربوط به خروجی مرحله‌ی نخست روش
پیشنهادی ماست در دسترس خواننده قرار گرفته است.
}
%%%%%%%%%%%%%%%%%%%%%%%%%%%%%
\clearpage
\section{مطالب اضافی}
این مطالب اضافی است که باید به نحوی در متن جای بگیرند.

\section{تعریف، تاریخچه و خواص بوروفن و
بوروفن هیدروژنه
}
بورافن یک نانومواد دوبعدی \lr{(2D)} است که توسط اتم‌های بور تشکیل شده و به گروه عنصری \lr{3A} در جدول تناوبی تعلق دارد. بور به عنوان تنها عنصر نامتال و نیمه‌رسانای گروه خود شناخته می‌شود. بور دارای پیکربندی الکترونی $1s^1 2s^2 2p^1$ است و اوربیتال هیبریدیزه $sp^2$ را دارا می‌باشد. اتم‌های بور از طریق ویژگی‌های اوربیتال، بسیار شبیه اتم‌های کربن عمل می‌کنند.
تولید مواد \lr{2D} جدید از عناصر \lr{3A، 4A} و \lr{5A} از جدول تناوبی عناصر پس از کشف گرافن توجه زیادی به خود جلب کرد. ویژگی‌های آن‌ها ارزش صفت‌های متعالی دارد. ساختار تک عنصری این امکان را فراهم می‌کند که به راحتی و سریع در سیستم‌های زیستی متابولیسم شود. سنتز ساده می‌تواند ارائه شود زیرا که آن‌ها به صورت تک عنصری تولید می‌شوند. این نانومواد \lr{2D} می‌توانند در برنامه‌های رسانا به کار روند زیرا که عناصر اصلی آن‌ها نیمه‌رسانا هستند. به دلیل ویژگی‌های ذکر شده قبلی، نانومواد \lr{2D} تک عنصری ممکن است برجسته باشند. نانومواد \lr{2D} پیشنهاد شده ویژگی‌هایی مانند مساحت سطح بزرگ و فعالیت‌های فیزیوشیمیایی دارند که اطمینان می‌دهد که این مواد برای استفاده در سوپرکاپاسیتورها، ذخیره هیدروژن، بیوتصویربری، حمل دارو، مواد شبیه به بیولوژی، بیوسنسورها و سایر دستگاه‌های الکترونیکی مناسب باشند. نانومواد‌های کشف شده براساس اتم‌هایی که از آن‌ها ساخته شده‌اند نام‌گذاری شده‌اند، مانند: سیلیسن، گرمانن، استانن، فسفورن، آرسنن، آنتیمونن، بیسموتن و بورافن. تعداد زیادی ساختار مختلف بورافن وجود دارد. این ساختارها براساس الگوهای منحصر به فردشان نام‌گذاری شده‌اند. دسته‌بندی ساختارهای خوشه‌ای بورون می‌تواند به سه دسته زیر تقسیم شود: 1) صفحه شش‌ضلعی تحریف‌شده \lr{(DH)}، 2) صفحه مثلثی خمیده \lr{(BT)} و 3) صفحه مخلوط مثلثی-شش‌ضلعی \lr{(MTH)}. صفحه‌های شش‌ضلعی تحریف‌شده توسط واحدهای شش‌ضلعی مانند گرافن به عنوان ساختارهای نیمه‌صفحه‌ای تقریبی تشکیل شده‌اند. صفحه‌های مثلثی خمیده توسط تکرار ساختارهای مخروطی $B_7$ ساخته شده‌اند در حالی که صفحه‌های مخلوط مثلثی-شش‌ضلعی توسط ساختارهای شش‌ضلعی خالی و الگوهای مثلثی ساخته شده‌اند. در یک اصل نام‌گذاری دیگر برای مواد \lr{2D} تک عنصری، آن‌ها می‌توانند براساس شماره‌های هماهنگی خاص خود دسته‌بندی شوند؛ (هنگامی که \lr{CN} برابر 5 یا 6 است) نوع $\alpha$، (هنگامی که \lr{CN} برابر 4، 5 یا 6 است) نوع $\beta$، (هنگامی که CN برابر 4 یا 5 است) نوع $\chi$، (هنگامی که \lr{CN} یک عدد است) نوع $\delta$ و (هنگامی که \lr{CN} برابر 3، 4 یا 5 است) نوع $\psi$. صفحه‌های \lr{DH} و \lr{BT} می‌توانند در زیر دسته‌بندی شوند؛ به عبارت دیگر، صفحه‌های \lr{MTH} می‌توانند توسط هر صفحه‌ای به جز صفحه‌های نوع $\delta$ دسته‌بندی شوند.

تفاوت‌ها در ترکیب و الگو منجر به ویژگی‌های متمایز می‌شود. به عنوان مثال؛ بورافن نوع $\beta_{12}$ ویژگی‌های ناهمجهز را نشان می‌دهد \lr{(Zhou, et al. 2017)} در حالی که نوع $\alpha$ ویژگی‌های همجهز را نشان می‌دهد \lr{(Xiao, et al. 2017)}. فاز \lr{2-Pmmn} اتم‌های بور را در جهت یک زیگ‌زاگ قرار داده است. با این حال، این فاز شاخص پواسون منفی دارد. انیسوتروپی بالا و مدول یانگ بزرگتر از گرافن قابل مشاهده است \lr{(Wang, et al. 2019)}. وجود نانوشیت‌های بورافن با یک مطالعه نظری آغاز شد \lr{(Boustani 1995)} که موفق به پیش‌بینی خوشه‌های بورون به شکل نیمه‌صفحه‌ای تقریبی شده است. این مقاله توسط مطالعات آینده دنبال شده است. \lr{Kuntsmann} و \lr{Ouandt (Kunstmann and Quandt 2006)} ساختارهای خمیده اصلی نانوشیت‌های بورون \lr{2D} را با اصل آوفباو توضیح دادند. در دوره‌ای که ساختار خمیده به عنوان پایدارترین خوشه بورون شناخته می‌شود، مطالعه دیگری \lr{(Tang and Beigi 2007)} پیش‌بینی کرد که ساختار نوع $\alpha$ نوعی از برگ بورون با انرژی پایین‌تر است. مطالعه الهام‌بخش بعدی \lr{(Wang, Zhang and Lin 2014)} مفهوم بورافن را توصیف کرد و اثبات کرد که ساختارهای بورافن با حفره‌های شش‌ضلعی مانند $B_{36}$ بین دیگران بسیار پایدار هستند.

توسط نرم‌افزارهای پژوهشی بر روی \lr{Borophene}، از یک سطح بورون ساخت کرده شد، اما تا سال 2015 مرکزی تولید بورون ساخته نشد[2] بورون ساختارهای پر از انرژی بونده شده توسط ارتباط بسیار قوی بین اتموم[1] و انیسوتروپی بالای سرعت انرژی با ترکیب بسیار بالا در دوران بالای 1000 درجه سانتیگرادی به دلیل کوبش بسیار خوب بین اتموم[1] بورون ساختارهای پر از انرژی بونده شده توسط ارتباط بسیار قوی بین اتموم[1] و انیسوتروپی بالای سرعت انرژی با ترکیب بسیار بالا در دوران بالای 1000 درجه سانتیگرادی به دلیل کوبش بسیار خوب بین اتموم[1] بورون ساختارهای پر از انرژی بونده شده توسط ارتباط بسیار قوی بین اتموم[1 توسط دو راه اصلی، بالای و زمینی \lr{(Bottom-up و Top-down)}، بورون ساخته شده است[3] برای تنظیم و اصلاح تصویری از متن زیر، به این موردهای زیر پرداخته شده است:

1. استفاده از کلمه "بورون" بیشتر و از کلمه \lr{"Borophene"} کمتر استفاده شود.
2. استفاده از کلمه "ساختار" بیشتر و از کلمه "شیشه" کمتر استفاده شود.
3. استفاده از کلمه "نوع" بیشتر و از کلمه "ترکیب" کمتر استفاده شود.
4. استفاده از کلمه "ساختار" برای نشان دادن بورون ساختارهای پر از انرژی و انیسوتروپی بالا و از کلمه "شیشه" برای نشان دادن بورون ساختارهای پر از انرژی با ترکیب بسیار بالا و سرعت انرژی بالا استفاده شود.

تصویری از تغییرات نامبوز شده است:

به وسیله دو راه اصلی، بالای و زمینی \lr{(Bottom-up و Top-down)}، بورون ساخته شده است[3]

بورون ساختارهای پر از انرژی بونده شده توسط ارتباط بسیار قوی بین اتموم[1] و انیسوتروپی بالای سرعت انرژی با ترکیب بسیار بالا در دوران بالای 1000 درجه سانتیگرادی به دلیل کوبش بسیار خوب بین اتموم[1] بورون ساختارهای پر از انرژی بونده شده توسط ارتباط بسیار قوی بین اتموم[1 به وسیله دو راه اصلی، بالای و

توسیع تصویر بروفن \lr{(Borophene)} به عنوان مواد نانو یک دسته ای از انرژی و تناوبی ترین مواد 2D است. بروفن به عنوان یک مواد با انرژی انرژی بسیار کم و سرعت انرژی بالا در اثر کمترین تغییرات در شرایط ساخت و ترکیب کننده با انرژی بالاتری از گرافن و ترکیب بسیار بالاتری از گرافن در مورد ترکیبات بروفن با پودرهای بورن با انرژی بالا ارایه شده است[1][2][3][4].
