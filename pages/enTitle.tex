% !TeX root=main.tex
% در این فایل، عنوان پایان‌نامه، مشخصات خود و چکیده پایان‌نامه را به انگلیسی، وارد کنید.

%%%%%%%%%%%%%%%%%%%%%%%%%%%%%%%%%%%%
\baselineskip=.6cm
\begin{latin}
\latinuniversity{Iran University of Science and Technology}
\latinfaculty{Physics Department}
\latinsubject{Ph.D}
\latinfield{Condensed Matter}
\latintitle{Valley polarized electronic and spintronic transport in 2D boron sheets}
\firstlatinsupervisor{Dr. Amirhossein Ahmadkhan Kordbacheh}
%\secondlatinsupervisor{Second Supervisor}
%\firstlatinadvisor{Dr. Einollah Khanjari}
% \secondlatinadvisor{Dr. X}
\projectLabel{Thesis}
 
\degree{Ph.D}
\degree{دکتری}

\latinname{\lr{Erfan}}
\latinsurname{\lr{Nikan}}
\latinthesisdate{February 2024}
\latinkeywords{Borophene, NEGF, }
\en-abstract{
In this thesis, we investigate the conductance properties of β12-borophene nanoribbons (BNRs) with single edge defects and weak disorder using the tight-binding model and Green function technique. We find that single edge defects lead to quasi-localized states near the defects, resulting in conductance dips at anti-resonance energies of edge states and defect-induced quasi-localized states. The influence of a single vacancy on the conductance is found to be larger than that of two vacancies, due to sublattice symmetry breaking. A weak disorder at the edges of the nanoribbons has a slight impact on conductance, but as the disorder potential increases, more localized eigenstates are created, leading to a conductor-to-semiconductor transition due to Anderson localization. Furthermore, we present a theoretical analysis of spin-dependent transport and spin-transport torque for a borophene-based ferromagnetic/normal/ferromagnetic junction. We focus on BNRs as a basis for spin valve numerical calculations for conduction in both parallel and antiparallel configurations of lead magnetizations, magnetoresistance (MR), and spin transfer torque (STT). Our results show that the conductance is always larger than e2/h for the parallel configuration, while it becomes zero in specific energy ranges for the antiparallel configuration. A perfect MR plateau is found for the junction in the absence of disorder, indicating its potential as an excellent spin valve candidate. The spin transfer torque (STT) per unit bias voltage decreases significantly in the Dirac point energy, and a sinusoidal oscillatory pattern can be observed in the STT at unit bias voltage versus the angle between the magnetizations of two electrodes, which amplifies as the ferromagnetic magnetization increases. Borophene has unique properties, including low density, high hardness, heat resistance, and electrical conductance, making it a promising candidate for spintronic applications. Our analysis provides a comprehensive understanding of the spin-dependent properties of borophene and its potential applications in spintronics.
}
\latinfirstPage
\end{latin}
