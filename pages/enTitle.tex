% !TeX root=main.tex
% در این فایل، عنوان پایان‌نامه، مشخصات خود و چکیده پایان‌نامه را به انگلیسی، وارد کنید.

%%%%%%%%%%%%%%%%%%%%%%%%%%%%%%%%%%%%
\baselineskip=.6cm
\begin{latin}
\latinuniversity{Iran University of Science and Technology}
\latinfaculty{Physics Department}
\latinsubject{Ph.D}
\latinfield{Condensed Matter}
\latintitle{Valley polarized electronic and spintronic transport in 2D boron sheets}
\firstlatinsupervisor{Dr. Amirhossein Ahmadkhan Kordbacheh}
%\secondlatinsupervisor{Second Supervisor}
%\firstlatinadvisor{Dr. Einollah Khanjari}
% \secondlatinadvisor{Dr. X}
\projectLabel{Thesis}
 
\degree{Ph.D}
\degree{دکتری}

\latinname{\lr{Erfan}}
\latinsurname{\lr{Nikan}}
\latinthesisdate{February 2024}
\latinkeywords{Borophene, NEGF, }
\en-abstract{
Exploring community structure is an appealing problem that has been 
drawing much attention in the recent years. One serious problem regarding 
many community detection methods is that the complete information of real-world
networks usually may not be available most of the time, also considering the
dynamic nature of such networks(e.g. web pages, collaboration networks and
users friendships on social networks), it is most probable possibility that
one could detect community structure from a certain source vertex with limited
knowledge of the entire network. The existing approaches can do well in measuring
the community quality, Nevertheless they are largely dependent on source vertex
chosen for the process. Additionally, using unsuitable seed vertices may lead to
finding of low quality or erroneous communities for output of many of the algorithms.
This paper proposes a method to find better source vertices to be used as seeds to
construct community structures locally. Inspired by the fact that many gargantuan 
real-world networks and respectively their graphs contain a myriad of lightly
connected vertices, we explore community structure heuristically by giving
priority to vertices which have a high number of links pertaining to the
core structure of the network. Experimental results prove that our method
can perform effectively for finding high quality seed vertices.
}
\latinfirstPage
\end{latin}
