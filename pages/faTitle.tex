% !TeX root=main.tex
% در این فایل، عنوان پایان‌نامه، مشخصات خود، متن تقدیمی‌، ستایش، سپاس‌گزاری و چکیده پایان‌نامه را به فارسی، وارد کنید.
% توجه داشته باشید که جدول حاوی مشخصات پروژه/پایان‌نامه/رساله و همچنین، مشخصات داخل آن، به طور خودکار، درج می‌شود.
%%%%%%%%%%%%%%%%%%%%%%%%%%%%%%%%%%%%
% دانشگاه خود را وارد کنید
\university{علم و صنعت ایران}
% دانشکده، آموزشکده و یا پژوهشکده  خود را وارد کنید
\faculty{دانشکده فیزیک}
% گروه آموزشی خود را وارد کنید
\department{گروه ماده چگال}
% گروه آموزشی خود را وارد کنید
\subject{فیزیک}
% گرایش خود را وارد کنید
\field{ماده چگال}
% عنوان پایان‌نامه را وارد کنید
\title{ترابرد اسپین و بار الکترونی در صفحات دوبعدي بورفین}
% نام استاد(ان) راهنما را وارد کنید
\firstsupervisor{دکتر امیرحسین احمدخان کردبچه}
%\secondsupervisor{استاد راهنمای دوم}
% نام استاد(دان) مشاور را وارد کنید. چنانچه استاد مشاور ندارید، دستور پایین را غیرفعال کنید.
%\firstadvisor{استاد مشاور اول}
%\secondadvisor{استاد مشاور دوم}
% نام دانشجو را وارد کنید
\name{عرفان}
% نام خانوادگی دانشجو را وارد کنید
\surname{نیکان}
% شماره دانشجویی دانشجو را وارد کنید
\studentID{9691117}
% تاریخ پایان‌نامه را وارد کنید
\thesisdate{بهمن  1402}
% به صورت پیش‌فرض برای پایان‌نامه‌های کارشناسی تا دکترا به ترتیب از عبارات «پروژه»، «پایان‌نامه» و »رساله» استفاده می‌شود؛ اگر  نمی‌پسندید هر عنوانی را که مایلید در دستور زیر قرار داده و آنرا از حالت توضیح خارج کنید.
\projectLabel{رساله}

% به صورت پیش‌فرض برای عناوین مقاطع تحصیلی کارشناسی تا دکترا به ترتیب از عبارات «کارشناسی»، «کارشناسی ارشد» و »دکترا» استفاده می‌شود؛ اگر  نمی‌پسندید هر عنوانی را که مایلید در دستور زیر قرار داده و آنرا از حالت توضیح خارج کنید.
\degree{دکتری}

\firstPage
\besmPage
\davaranPage

%\vspace{.5cm}
% در این قسمت اسامی اساتید راهنما، مشاور و داور باید به صورت دستی وارد شوند
%\renewcommand{\arraystretch}{1.2}
\begin{center}
\begin{tabular}{| p{8mm} | p{18mm} | p{.17\textwidth} |p{14mm}|p{.2\textwidth}|c|}
\hline
ردیف	& سمت & نام و نام خانوادگی & مرتبه \newline دانشگاهی &	دانشگاه یا مؤسسه & امضــــــــــــا\\
\hline
۱  & استاد راهنما & دکتر \newline امیرحسین احمدخان کردبچه
& دانشیار & دانشگاه \newline علم و صنعت ایران &  \\
\hline
۲ & استاد داور \newline داخلی	 & دکتر \newline ادریس فیض
آبادي  & استاد & 
دانشگاه  \newline علم ‌و صنعت ایران & \\
\hline
3 & استاد داور \newline داخلی	 & دکتر \newline افشین نمیرانیان  & دانشیار & 
دانشگاه  \newline علم ‌و صنعت ایران & \\
\hline
4 &	استاد داور \newline  خارجی  & دکتر 
\newline علی اصغر شکري
& استاد & دانشگاه \newline  پیام نور تهران  & \\
\hline
5 &	استاد داور \newline  خارجی  & دکتر 
\newline  مهران باقري 
& استادیار & دانشگاه \newline  شهید بهشتی  & \\
\hline
\end{tabular}
\end{center}

\esalatPage
\mojavezPage

 \newpage
\thispagestyle{empty}
{\Large تقدیم به:}\\
\begin{flushleft}
{\huge
%\lr{Pein}\\
%\vspace{7mm}
%و \\
%\vspace{7mm}
\textbf{
پدر و مادرم.
}}
\end{flushleft}

% 
% % سپاس‌گزاری
% \begin{acknowledgementpage}
% سپاس خداوندگار حکیم را که با لطف بی‌کران خود، آدمی را زیور عقل آراست.
% 
% 
% در آغاز وظیفه‌  خود  می‌دانم از زحمات بی‌دریغ استاد  راهنمای خود،  جناب آقای دکتر ...، صمیمانه تشکر و  قدردانی کنم  که قطعاً بدون راهنمایی‌های ارزنده‌  ایشان، این مجموعه  به انجام  نمی‌رسید.
% 
% از جناب  آقای  دکتر ...   که زحمت  مطالعه و مشاوره‌  این رساله را تقبل  فرمودند و در آماده سازی  این رساله، به نحو احسن اینجانب را مورد راهنمایی قرار دادند، کمال امتنان را دارم.
% 
% همچنین لازم می‌دانم از پدید آورندگان بسته زی‌پرشین، مخصوصاً جناب آقای  وفا خلیقی، که این پایان‌نامه با استفاده از این بسته، آماده شده است و همه دوستانمان در گروه پارسی‌لاتک کمال قدردانی را داشته باشم.
% 
%  در پایان، بوسه می‌زنم بر دستان خداوندگاران مهر و مهربانی، پدر و مادر عزیزم و بعد از خدا، ستایش می‌کنم وجود مقدس‌شان را و تشکر می‌کنم از خانواده عزیزم به پاس عاطفه سرشار و گرمای امیدبخش وجودشان، که بهترین پشتیبان من بودند.
% % با استفاده از دستور زیر، امضای شما، به طور خودکار، درج می‌شود.
% \signature 
% \end{acknowledgementpage}
%%%%%%%%%%%%%%%%%%%%%%%%%%%%%%%%%%%%
% کلمات کلیدی پایان‌نامه را وارد کنید
\keywords{یافتن اجتماعات، شبکه‌های پیچیده، الگوریتم‌های گراف}
%چکیده پایان‌نامه را وارد کنید، برای ایجاد پاراگراف جدید از \\ استفاده کنید. اگر خط خالی دشته باشید، خطا خواهید گرفت.
\fa-abstract{
ساختار اجتماع خصوصیتی فراگیر در شبکه‌های پیچیده است. مساله‌ی یافتن اجتماعات در این شبکه‌ها جزو‌ مسایل مورد توجه محققین در
چند سال اخیر بوده است.
اجتماع مجموعه‌ای از گره‌های گراف می‌باشد
که در عین حالی که با هم دارای اتصالات زیادی می‌باشند از بقیه گراف به خوبی مجزا هستند.
گراف‌ ‌شبکه‌های پیچیده دارای خواص ساختاری مانند کوتاهی فاصله‌ دو گره‌ دلخواه هستند
که آن‌ها را از گراف‌های تصادفی مطالعه شده در گذشته متمایز می‌نماید.
از طرفی الگوریتم‌های یافتن اجتماعات را می‌توان به دو دسته‌ی الگوریتم‌های محلی و سراسری تقسیم کرد.
یکی از چالش‌های الگوریتم‌های محلی انتخاب گره‌های دانه است.
در این پایان‌نامه روشی برای یافتن اجتماعات روی گراف شبکه‌های
پیچیده به کمک انتخاب دانه‌های مرغوب و بسط این گره‌ها توسط یک الگوریتم محلی ارایه شده است.
 روش‌ پیشنهادی ۳ مرحله دارد،
 در مرحله‌ی نخست گره‌های گراف ورودی به کمک یک استراتژی حریصانه به چندین افراز تقسیم می‌شوند.
 این افراز‌ها نشان دهنده‌ی اجتماعات اولیه گراف‌ ورودی خواهند بود.
 در گام دوم درون هر زیرگراف حاصل از گره‌های درون یک افراز و اتصالات میان گره‌های آن
 ‌  به دنبال گره‌هایی که به احتمال زیادی به خوبی در بطن یک اجتماع واقع شده‌اند می‌گردیم.
  در این مرحله در هر زیرگراف به صورت موازی به کمک بررسی همسایگی گره‌های
  با درجه‌ بالا، گره‌هایی را که می‌توانند به خوبی نشانگر اجتماع خود باشند را به عنوان گره‌ دانه برمی‌گزینیم.
 در گام آخر اجتماعاتی که هر گره دانه در آن قرار گرفته است را به کمک الگوریتمی که بر پایه محاسبه‌ی
 بردار
 \lr{Personalized Pagerank}
عمل می‌کند، می‌یابیم.
 برای آزمایش کیفیت اجتماعات خروجی
 روش پیشنهادی خود، از ۲۲ گراف استاندارد که در ۷ دسته مختلف از شبکه‌های پیچیده قرار می‌گیرند استفاده نموده ایم.
 برای سنجش کیفیت اجتماعات روش‌ پیشنهادی میزان ۵ معیار مختلف سنجش اجتماعات برای خروجی روش ‌ما و ۳ روش دیگر آزمایش شده است.
 کیفیت اجتماعات روش پیشنهادی برای ۲ معیار از ۵ معیار بسیار بهتر از خروجی دیگر روش‌هاست، برای ۳ معیار دیگر هم عملکرد
 روش‌ پیشنهادی بسیار شبیه عملکرد روش با بهترین خروجی بوده است.
نتایج‌ نشان‌ می‌دهد که نباید تنها درجه‌ یک گره‌ را به عنوان معیار صرف دانه بودن آن انتخاب کرد.
}

\abstractPage

\newpage\clearpage