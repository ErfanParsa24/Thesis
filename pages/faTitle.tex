% !TeX root=main.tex
% در این فایل، عنوان پایان‌نامه، مشخصات خود، متن تقدیمی‌، ستایش، سپاس‌گزاری و چکیده پایان‌نامه را به فارسی، وارد کنید.
% توجه داشته باشید که جدول حاوی مشخصات پروژه/پایان‌نامه/رساله و همچنین، مشخصات داخل آن، به طور خودکار، درج می‌شود.
%%%%%%%%%%%%%%%%%%%%%%%%%%%%%%%%%%%%
% دانشگاه خود را وارد کنید
\university{علم و صنعت ایران}
% دانشکده، آموزشکده و یا پژوهشکده  خود را وارد کنید
\faculty{دانشکده فیزیک}
% گروه آموزشی خود را وارد کنید
\department{گروه ماده چگال}
% گروه آموزشی خود را وارد کنید
\subject{فیزیک}
% گرایش خود را وارد کنید
\field{ماده چگال}
% عنوان پایان‌نامه را وارد کنید
\title{ترابرد اسپین و بار الکترونی در صفحات دوبعدي بورفین}
% نام استاد(ان) راهنما را وارد کنید
\firstsupervisor{دکتر امیرحسین احمدخان کردبچه}
%\secondsupervisor{استاد راهنمای دوم}
% نام استاد(دان) مشاور را وارد کنید. چنانچه استاد مشاور ندارید، دستور پایین را غیرفعال کنید.
%\firstadvisor{استاد مشاور اول}
%\secondadvisor{استاد مشاور دوم}
% نام دانشجو را وارد کنید
\name{عرفان}
% نام خانوادگی دانشجو را وارد کنید
\surname{نیکان}
% شماره دانشجویی دانشجو را وارد کنید
\studentID{9691117}
% تاریخ پایان‌نامه را وارد کنید
\thesisdate{اردیبهشت  1403}
% به صورت پیش‌فرض برای پایان‌نامه‌های کارشناسی تا دکترا به ترتیب از عبارات «پروژه»، «پایان‌نامه» و »رساله» استفاده می‌شود؛ اگر  نمی‌پسندید هر عنوانی را که مایلید در دستور زیر قرار داده و آنرا از حالت توضیح خارج کنید.
\projectLabel{رساله}

% به صورت پیش‌فرض برای عناوین مقاطع تحصیلی کارشناسی تا دکترا به ترتیب از عبارات «کارشناسی»، «کارشناسی ارشد» و »دکترا» استفاده می‌شود؛ اگر  نمی‌پسندید هر عنوانی را که مایلید در دستور زیر قرار داده و آنرا از حالت توضیح خارج کنید.
\degree{دکتری}

\firstPage
\besmPage
\davaranPage

%\vspace{.5cm}
% در این قسمت اسامی اساتید راهنما، مشاور و داور باید به صورت دستی وارد شوند
%\renewcommand{\arraystretch}{1.2}
\begin{center}
\begin{tabular}{| p{8mm} | p{18mm} | p{.17\textwidth} |p{14mm}|p{.2\textwidth}|c|}
\hline
ردیف	& سمت & نام و نام خانوادگی & مرتبه \newline دانشگاهی &	دانشگاه یا مؤسسه & امضــــــــــــا\\
\hline
۱  & استاد راهنما & دکتر \newline امیرحسین احمدخان کردبچه
& دانشیار & دانشگاه \newline علم و صنعت ایران &  \\
\hline
۲ & استاد داور \newline داخلی	 & دکتر \newline ادریس فیض
آبادي  & استاد & 
دانشگاه  \newline علم ‌و صنعت ایران & \\
\hline
3 & استاد داور \newline داخلی	 & دکتر \newline افشین نمیرانیان  & دانشیار & 
دانشگاه  \newline علم ‌و صنعت ایران & \\
\hline
4 &	استاد داور \newline  خارجی  & دکتر 
\newline علی اصغر شکري
& استاد & دانشگاه \newline  پیام نور تهران  & \\
\hline
5 &	استاد داور \newline  خارجی  & دکتر 
\newline  مهران باقري 
& استادیار & دانشگاه \newline  شهید بهشتی  & \\
\hline
\end{tabular}
\end{center}

\esalatPage
\mojavezPage

 \newpage
\thispagestyle{empty}
{\Large تقدیم به:}\\
\begin{flushleft}
{\huge
%\lr{Pein}\\
%\vspace{7mm}
%و \\
%\vspace{7mm}
\textbf{
پدر و مادرم.
}}
\end{flushleft}

% 
% % سپاس‌گزاری
% \begin{acknowledgementpage}
% سپاس خداوندگار حکیم را که با لطف بی‌کران خود، آدمی را زیور عقل آراست.
% 
% 
% در آغاز وظیفه‌  خود  می‌دانم از زحمات بی‌دریغ استاد  راهنمای خود،  جناب آقای دکتر ...، صمیمانه تشکر و  قدردانی کنم  که قطعاً بدون راهنمایی‌های ارزنده‌  ایشان، این مجموعه  به انجام  نمی‌رسید.
% 
% از جناب  آقای  دکتر ...   که زحمت  مطالعه و مشاوره‌  این رساله را تقبل  فرمودند و در آماده سازی  این رساله، به نحو احسن اینجانب را مورد راهنمایی قرار دادند، کمال امتنان را دارم.
% 
% همچنین لازم می‌دانم از پدید آورندگان بسته زی‌پرشین، مخصوصاً جناب آقای  وفا خلیقی، که این پایان‌نامه با استفاده از این بسته، آماده شده است و همه دوستانمان در گروه پارسی‌لاتک کمال قدردانی را داشته باشم.
% 
%  در پایان، بوسه می‌زنم بر دستان خداوندگاران مهر و مهربانی، پدر و مادر عزیزم و بعد از خدا، ستایش می‌کنم وجود مقدس‌شان را و تشکر می‌کنم از خانواده عزیزم به پاس عاطفه سرشار و گرمای امیدبخش وجودشان، که بهترین پشتیبان من بودند.
% % با استفاده از دستور زیر، امضای شما، به طور خودکار، درج می‌شود.
% \signature 
% \end{acknowledgementpage}
%%%%%%%%%%%%%%%%%%%%%%%%%%%%%%%%%%%%
% کلمات کلیدی پایان‌نامه را وارد کنید
\keywords{بوروفین،اسپین‌ترونیک،تابع‌گرین، ترابرد،مغناطومقاومت}
%چکیده پایان‌نامه را وارد کنید، برای ایجاد پاراگراف جدید از \\ استفاده کنید. اگر خط خالی دشته باشید، خطا خواهید گرفت.
\fa-abstract{
    در این پایان نامه، ما خواص رسانایی نانوروبان های $\beta_{12}$-بوروفن (\lr{BNRs}) با نقص های تک لبه و اختلال ضعیف را با استفاده از مدل اتصال محکم و تکنیک تابع گرین بررسی می کنیم. ما متوجه شدیم که نقص‌های تک لبه منجر به حالت‌های شبه موضعی در نزدیکی نقص‌ها می‌شود، که منجر به کاهش رسانایی در انرژی‌های ضد تشدید حالت‌های لبه و حالت‌های شبه موضعی ناشی از نقص می‌شود. به دلیل شکستن تقارن زیرشبکه، تأثیر یک جای خالی روی رسانایی بزرگتر از دو جای خالی است. یک اختلال ضعیف در لبه‌های نانوروبان‌ها تأثیر جزئی بر رسانایی دارد، اما با افزایش پتانسیل اختلال، حالت‌های ویژه محلی‌تر ایجاد می‌شود که به دلیل محلی‌سازی اندرسون منجر به انتقال رسانا به نیمه‌رسانا می‌شود. علاوه بر این، ما یک تحلیل نظری از انتقال وابسته به اسپین و گشتاور حمل و نقل اسپین برای یک اتصال فرومغناطیسی / نرمال / فرومغناطیسی مبتنی بر بوروفن ارائه می‌کنیم. ما بر روی \lr{BNR}ها به عنوان مبنایی برای محاسبات عددی دریچه چرخشی برای هدایت در هر دو پیکربندی موازی و ضد موازی مغناطیس‌های سرب، مقاومت مغناطیسی (\lr{MR}) و گشتاور انتقال اسپین (\lr{STT}) تمرکز می‌کنیم. نتایج ما نشان می‌دهد که رسانایی برای پیکربندی موازی همیشه بزرگ‌تر از $e^2/h$ است، در حالی که در محدوده انرژی خاص برای پیکربندی ضد موازی صفر می‌شود. یک فلات \lr{MR} کامل برای اتصال در غیاب بی نظمی یافت می شود، که نشان دهنده پتانسیل آن به عنوان کاندیدای عالی دریچه چرخشی است. گشتاور انتقال اسپین (\lr{STT}) در واحد ولتاژ بایاس به طور قابل توجهی در انرژی نقطه دیراک کاهش می یابد و یک الگوی نوسانی سینوسی را می توان در \lr{STT} در ولتاژ بایاس واحد در مقابل زاویه بین مغناطش دو الکترود مشاهده کرد که به عنوان مغناطیس فرومغناطیسی تقویت می شود. افزایش. بوروفن دارای خواص منحصر به فردی از جمله چگالی کم، سختی بالا، مقاومت در برابر حرارت و رسانایی الکتریکی است که آن را به یک نامزد امیدوارکننده برای کاربردهای اسپینترونیک تبدیل می کند. تجزیه و تحلیل ما درک جامعی از خواص بوروفن وابسته به اسپین و کاربردهای بالقوه آن در اسپینترونیک ارائه می دهد.
}

\abstractPage

\newpage\clearpage